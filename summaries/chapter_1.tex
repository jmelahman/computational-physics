\documentclass[10pt]{article}
\usepackage[total={6in,9.25in}]{geometry}
\usepackage[utf8]{inputenc}
\usepackage{listings}
\usepackage{hyperref}
\usepackage{graphicx}
\usepackage{amsmath}
\usepackage{amsfonts}
\usepackage{amssymb}
\usepackage{caption}
\usepackage[framed,numbered]{mcode}
\lstset{language=Python}
\author{Jamison Lahman}
\title{Computational Physics: Chapter 1}
\begin{document}

\begin{flushright}Jamison Lahman (\href{mailto:lahmanja@gmail.com}{lahmanja@gmail.com}) \\
\href{https://www.amazon.com/Computational-Physics-Fortran-Steven-Koonin/dp/0201386232}{Computational Physics: FORTRAN Version} \\
by Steven E. Koonin and Dawn C. Meredith\\
Chapter 01 Exercises | July 7th, 2018 \\
\end{flushright}
\textbf{Exercise 1.1:} Using any function for which you can evaluate the derivatives analytically, investigate the accuracy of the formulas in Table 1\footnote{Table 1.2 from \href{https://www.amazon.com/Computational-Physics-Fortran-Steven-Koonin/dp/0201386232}{Computational Physics: FORTRAN Version}} for various values of \textit{h}. \\
\\
\textbf{Summary:} To approximate the derivative numerically, we will utilize five equations based off the Maclaurin Series. The Maclaurin Series of function, \textit{f(x)}, is given by the equation,
\begin{equation}
f(x) = f_0 +x f_0'+ \frac{x^2f_0''}{2!} + \frac{x^nf_0^(n)}{n!}.
\end{equation}
Solving the above equation for \textit{f'} yields,
\begin{equation}
f_0' = \frac{f(x)-f_0}{x}+\frac{-1}{x}\left(\frac{x^2f_0''}{2!} \right)+ \frac{-1}{x} \left(\frac{x^nf_0^{(n)}}{n!}\right) =  \frac{f(x)-f_0}{x}  + \mathcal{O}(x).
\end{equation}
The above equation assumes the function is accurately described by a linear function over the interval $(0,h)$. This gives the first of the '2-point' methods, 
\begin{equation}
\label{eq:forward2}
f'(x) \approx \frac{f(x+h)-f(x)}{h},
\end{equation}
where \textit{x} is the value where the derivative is being evaluated and \textit{h} is the step size used. Using the same method as above but using a previous point produces the equation,
\begin{equation}
\label{eq:backward2}
f'(x) \approx \frac{f(x)-f(x-h)}{h}
\end{equation}
assumes the function is accurately described by a linear function over the interval $(-h,0)$. Notice each of these equations are accurate up to one order of \textit{h}. 

We can improve the accuracy by including more terms. Using the immediate values forwards and backwards allows for the terms with even-powered values of \textit{h} to cancel out. Eq \ref{eq:symmetric3} is the quadratic polynomial interpolation of the function over the two previous regions,
\begin{equation}
\label{eq:symmetric3}
f'=\frac{f(x+h)-f(x-h)}{2h} + \mathcal{O}(h^2) \approx \frac{f(x+h)-f(x-h)}{2h}.
\end{equation} 

The final two equations, the "4-point" and "5-point" formulas, are similarly found and include more interactions fowards and backwards to create higher-order approximations of the function over the given interval. They, along with their higher-order derivatives, are given in Table \ref{tab:differentiation}. The 5-point formula is the five-point stencil in one-dimension. \\
\begin{table}[!h]
	% Center the table
	\begin{center}
    % Title of the table
	\caption{4- and 5-point difference formulas for derivatives}
		\label{tab:differentiation}
		\begin{tabular}{ccc}
		% To create a horizontal line, type \hline
		\hline
		& 4-point & 5-point \\
		\hline
		$hf'$ & $\pm\frac{1}{6}(-2f_{\mp1} - 3f_{0} + 6f_{\pm 1} - f_{\pm 2})$ & $\frac{1}{12}(f_{-2} - 8f_{-1} + 8f_{1} -f_{2})$ \\
		$h^{2}f'' $ & $f_{-1} - 2f_{0} + f_{1}$ & $\frac{1}{12}(-f_{-2}+16f_{-1} - 30 f_{0} +16f_{1} -f_{2})$ \\
		$h^{3}f'''$ & $\pm(-f_{\mp 1} + 3 f_{0} - 3 f_{\pm 1} + f_{\pm 2 })$ & $\frac{1}{2}(-f_{-2} + 2f_{-1} - 2f_{1} + f_{2})$ \\
		$h^{4}f^{iv)} $ & ... & $f_{-2} - 4 f_{-1} + 6 f_{0} -4f_{1} + f_{2}$
		\end{tabular}
	\end{center}
\end{table} \\
\textbf{Solution:} The function we will approximating the derivative numerically is,
\[
F(x) = \sin(x).
\]
The analytical derivative of the sine function is known to be,
\[
\frac{d \sin(x)}{dx} = \cos(x).
\]
For a formal proof of the derivation, see \href{https://en.wikipedia.org/wiki/Differentiation_of_trigonometric_functions#Derivative_of_the_sine_function}{Wikipedia - derivative of the sine function}. The exact value of the derivative at $x=1$ (up to six decimal points) is $0.540302$. Table \ref{tab:diff_errors} shows the accuracy of the formulas above for various values of \textit{h}. \\

\begin{table}[!ht]
	% Center the table
	\begin{center}
    % Title of the table
	\caption{Error in evaluating the $d \sin /dx|_{x=1}=0.540302$}
		\label{tab:diff_errors}
		\begin{tabular}{cccccc}
		% To create a horizontal line, type \hline
		\hline
		h & Fwd 2-point Eq. \ref{eq:forward2} & Bkwd 2-point Eq. \ref{eq:backward2} & 3-point Eq. \ref{eq:symmetric3} & 4-point Tab. \ref{tab:differentiation} & 5-point Tab. \ref{tab:differentiation} \\
		\hline
0.50000&-0.228254&0.183789&-0.022233&-0.009499&-0.001093\\ 
0.20000&-0.087462&0.080272&-0.003595&-0.000586&-0.000029\\ 
0.10000&-0.042939&0.041138&-0.000900&-0.000072&-0.000002\\ 
0.05000&-0.021257&0.020807&-0.000225&-0.000009&-0.000000\\ 
0.02000&-0.008450&0.008378&-0.000036&-0.000001&-0.000000\\ 
0.01000&-0.004216&0.004198&-0.000009&-0.000000&-0.000000\\ 
0.00500&-0.002106&0.002101&-0.000002&-0.000000&-0.000000\\ 
0.00200&-0.000842&0.000841&-0.000000&-0.000000&-0.000000\\ 
0.00100&-0.000421&0.000421&-0.000000&-0.000000&-0.000000\\ 
0.00050&-0.000210&0.000210&-0.000000&-0.000000&-0.000000\\ 
0.00020&-0.000084&0.000084&-0.000000&-0.000000&-0.000000\\ 
0.00010&-0.000042&0.000042&-0.000000&0.000000&0.000000\\ 
0.00005&-0.000021&0.000021&-0.000000&0.000000&0.000000\\ 
0.00002&-0.000008&0.000008&-0.000000&-0.000000&0.000000\\ 
0.00001&-0.000004&0.000004&-0.000000&-0.000000&-0.000000\\ 
\hline
		\end{tabular}
	\end{center}
\end{table}
Python interpretations of the formulas from above\footnote{For the full python script responsible for Table \ref{tab:diff_errors}, see my \href{https://github.com/jmelahman/computational-physics-solutions/blob/master/exercises_python/exercise1_1.py}{GitHub}.},

\begin{lstlisting}
def forward2(x,h):
#Performs the forward 2-point method on the function defined by myFunc
#Input:  x -- independent variable
#        h -- step-size
#Output: ans -- dependent variable
    ans = (myFunc(x)-myFunc(x-h))/h
    return(ans)
\end{lstlisting}

\begin{lstlisting}
def backward2(x,h):
#Performs the backward 2-point method on the function defined by myFunc
#Input:  x -- independent variable
#        h -- step-size
#Output: ans -- dependent variable
    ans = (myFunc(x+h)-myFunc(x))/h
    return(ans)
\end{lstlisting}
\begin{lstlisting}
def symmetric3(x,h):
#Performs the symmetric 3-point method on the function defined by myFunc
#Input:  x -- independent variable
#        h -- step-size
#Output: ans -- dependent variable
    ans = (myFunc(x+h)-myFunc(x-h))/(2*h)
    return(ans)
\end{lstlisting}
\begin{lstlisting}
def symmetric4(x,h):
#Performs the symmetric 4-point method on the function defined by myFunc
#Input:  x -- independent variable
#        h -- step-size
#Output: ans -- dependent variable
    ans = (-2*myFunc(x-h)-3*myFunc(x)+6*myFunc(x+h)-myFunc(x+2*h))/(6*h)
    return(ans)
\end{lstlisting}
\begin{lstlisting}
def symmetric5(x,h):
#Performs the symmetric 5-point method on the function defined by myFunc
#Input:  x -- independent variable
#        h -- step-size
#Output: ans -- dependent variable
    ans = (myFunc(x-2*h)-8*myFunc(x-h)+8*myFunc(x+h)-myFunc(x+2*h))/(12*h)
    return(ans)
\end{lstlisting}
\newpage
\noindent
\textbf{Exercise 1.2:} Using any function whose definite integral you can compute analytically, investigate the accuracy of various quadrature methods for various values of \textit{h}. \\
\\
\textbf{Summary:}
\end{document}